\documentclass[20pt]{article}
\usepackage[margin=1in]{geometry}
\usepackage{amsfonts,amsmath,amssymb}
\usepackage[none]{hyphenat}
\usepackage{fancyhdr}
\parindent 0ex
\pagestyle{fancy}
\fancyhead{}
\fancyhead[L]{}
\renewcommand{\baselinestretch}{1.5}
\usepackage{graphicx}
\begin{document}

\begin{titlepage}
\begin{center}
\vspace*{3cm}
\huge{\textbf{Pell's Equation}}\\
\Large{\textbf{January 2022}}\\
\vfill
Yashashwi Singhania\\

\end{center}
\end{titlepage}
\section{Introduction}
Pell's Equations are special types of Diophantine equations(equations with integer solutions) of the form $x^2 - ny^2 =1$ where $n$ is a given positive \textit{nonsquare integer} and the integer solutions are sought for $x$ and $y$. On plotting these equations on a cartesian plane, we get a parabolic curve where $(\pm{1},0)$ is a trivial solution.\\
Lagrange proved that as long as $n$ is not a perfect square, there are infinite solutions to this diophantine equation.\\

First studies of the Pell's Equation was actually done in India.
The Indian mathematician and philosopher Brahmagupta found first integer solution to $92x^2 + 1=y^2$ in his \emph{Brahmasputasiddhanta}. Bhaskara II in the twelfth century Narayana Pandit in he 14th century both found a general solution to the Pell's Equation. \\

Now if you're motivated enough, try and find the primitive solution\footnote{smallest non-trivial $x$ and $y$} to $x^2 + 61y^2 = 1$.

\section{History}

As early as 400 BC in India and Greece, mathematicians studied the numbers arising from the n = 2 case of Pell's equation,\\
$x^2 + 2y^2 = \pm 1$\\
because of the connection of these equations to the square root of 2. Indeed, if x and y are positive integers satisfying this equation, then $\frac{x}{y}$ is an approximation of $\sqrt{2}$.\\
 Pythagoreans, and Proclus observed that in the opposite direction these numbers obeyed one of these two equations. Similarly, Baudhayana discovered that $x = 17, y = 12$ and $x = 577, y = 408$ are two solutions to the Pell's equation, and that $\frac{17}{12}$ and $\frac{577}{408}$ are very close approximations to $\sqrt{2}$.\\
 Later, Archimedes approximated the $\sqrt{3}$ by the rational number $\frac{1351}{780}$. Although he did not explain his methods, this approximation may be obtained in the same way, as a solution to Pell's equation.`\\
 
 \subsection{Exercise}
Prove that the only solutions to $x^2 - ny^2 = 1$ where $n$ is a square is $(\pm 1,0).$
\end{document}